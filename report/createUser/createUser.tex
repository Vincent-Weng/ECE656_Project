\documentclass[12pt]{scrbook}

\setkomafont{author}{\scshape}
\linespread{1}
\usepackage[margin=1in]{geometry} % 1 inch margins all around
\usepackage{graphicx} 
\graphicspath{ {../} }
\renewcommand\thesection{\arabic{section}}

% Merge 1: new package requirements
% COPY TO MASTER .tex FILE WHEN MERGING!
% COPY START
\usepackage{float}
\usepackage{listings}
\usepackage{inconsolata}

\lstset{
  basicstyle=\ttfamily,
}
% COPY END


\title{ECE656 Project Report}
\subtitle{Finding spam comments within the Yelp dataset}
\date{March 2018}
\author{Josh Reid\thanks{University of Waterloo}
\and Vincent Weng\footnotemark[1]}

% Merge 2: copy and attach the text to master .tex file
\begin{document}
% COPY START
\section{Part 2}
\subsection{Description of the problem}
    In real applications, the Yelp database is expected to be visited by different group of people, including customers (users), data analyst (special users), and developers. In this project, this is further divided into five catagories:
    \begin{enumerate}
      \item A casual user who uses the application to browse search results. These users do not need to have an account; hence, they cannot submit reviews.
      \item Critiques that use the application to browse results just like the casual user, but they also leave reviews for places they visit. A logged in user should only be provided enough priv- ileges to write the review.
      \item Business analysts can use the application to produce sales reports and may want to do special data mining and analysis. They cannot perform IUD (Insert/Update/Delete) opera- tions on the database but should have access to creating extra views on the database schema.
      \item Developers working with this database are able to create new tables and perform data cleaning and indexing. They are allowed to perform IUD operations on the database.
      \item The database admin who has full access over the database.
    \end{enumerate} 

    The principle of granting privilege is to guarantee that each group of people have sufficient permission in order to pretect the database. First, the list of all privileges in MySQL 5.7 are listed in Table \ref{tab:privileges}, from which we can choose levels for each user group\footnote{https://dev.mysql.com/doc/refman/5.7/en/grant.html}. 


\begin{table}
      \hspace{-1.5cm}
      \begin{tabular}{l|p{5.5in}} 
      \hline
      \textbf{Privilege} & \textbf{Meaning and Grantable Levels} \\
      \hline
      ALL [PRIVILEGES] & Grant all privileges at specified access level except GRANT OPTION and PROXY. \\
      ALTER & Enable use of ALTER TABLE. Levels: Global, database, table. \\
      ALTER ROUTINE & Enable stored routines to be altered or dropped. Levels: Global, database, procedure. \\
      CREATE & Enable database and table creation. Levels: Global, database, table. \\
      CREATE ROUTINE & Enable stored routine creation. Levels: Global, database. \\
      CREATE TABLESPACE & Enable tablespaces and log file groups to be created, altered, or dropped. Level: Global. \\
      CREATE USER & Enable use of CREATE USER, DROP USER, RENAME USER, and REVOKE ALL PRIVILEGES. Level: Global. \\
      CREATE VIEW & Enable views to be created or altered. Levels: Global, database, table. \\
      DELETE & Enable use of DELETE. Level: Global, database, table. \\
      DROP  & Enable databases, tables, and views to be dropped. Levels: Global, database, table. \\
      EVENT & Enable use of events for the Event Scheduler. Levels: Global, database. \\
      EXECUTE & Enable the user to execute stored routines. Levels: Global, database, table. \\
      FILE  & Enable the user to cause the server to read or write files. Level: Global. \\
      GRANT OPTION & Enable privileges to be granted to or removed from other accounts. Levels: Global, database, table, procedure, proxy. \\
      INDEX & Enable indexes to be created or dropped. Levels: Global, database, table. \\
      INSERT & Enable use of INSERT. Levels: Global, database, table, column. \\
      LOCK TABLES & Enable use of LOCK TABLES on tables for which you have the SELECT privilege. Levels: Global, database. \\
      PROCESS & Enable the user to see all processes with SHOW PROCESSLIST. Level: Global. \\
      PROXY & Enable user proxying. Level: From user to user. \\
      REFERENCES & Enable foreign key creation. Levels: Global, database, table, column. \\
      RELOAD & Enable use of FLUSH operations. Level: Global. \\
      REPLICATION CLIENT & Enable the user to ask where master or slave servers are. Level: Global. \\
      REPLICATION SLAVE & Enable replication slaves to read binary log events from the master. Level: Global. \\
      SELECT & Enable use of SELECT. Levels: Global, database, table, column. \\
      SHOW DATABASES & Enable SHOW DATABASES to show all databases. Level: Global. \\
      SHOW VIEW & Enable use of SHOW CREATE VIEW. Levels: Global, database, table. \\
      SHUTDOWN & Enable use of mysqladmin shutdown. Level: Global. \\
      SUPER & Enable use of other administrative operations such as CHANGE MASTER TO, KILL, PURGE BINARY LOGS, SET GLOBAL, and mysqladmin debug command. Level: Global. \\
      TRIGGER & Enable trigger operations. Levels: Global, database, table. \\
      UPDATE & Enable use of UPDATE. Levels: Global, database, table, column. \\
      USAGE & Synonym for  no privileges  \\
      \hline
      \end{tabular}%
    \caption{Privileges available in MySQL}
    \label{tab:privileges}%
  \end{table}%

\subsection{Group 1}
    For the first group of users, they only browse information about business, including their opening hours, stars, review, without signing in so they are normally not expected to write information into the database. In some cases, if the app allows some specific types of anonymous communications, such as marking a review as ``cool" or ``useful'' by a visitor, permission of making few modification on the number of these tags may be granted. However, in this project we assume that the user are not allowed to perform any operations except exploring. Hereby we only grant \texttt{SELECT} privilege to the first group of user, which we call \texttt{user1}:

    \begin{verbatim}
      DROP USER IF EXISTS 'user1'@'%';
      CREATE USER user1;
      GRANT SELECT ON yelp_db.* TO 'user1'@'%';
    \end{verbatim}
  
\subsection{Group 2}

    For the second type of user, they are different from casual users only by that they may leave reviews or tips on a business. They are logged-in users, so they can interact with other reviews or tips. Therefore, they are granted global \texttt{SELECT} privilege, \texttt{INSERT} on the review and tip table, \texttt{UPDATE} on certains columns in business table, and table-wise \texttt{UPDATE} on user table. The SQL query is shown as follows, similarly we call this \texttt{user2}:

    \begin{verbatim}
      DROP USER IF EXISTS 'user2'@'%';
      CREATE USER user2;
      GRANT SELECT ON yelp_db.* TO 'user2'@'%';
      GRANT INSERT ON yelp_db.review TO 'user2'@'%';
      GRANT INSERT ON yelp_db.tip TO 'user2'@'%';
      GRANT UPDATE (stars) ON yelp_db.business TO 'user2'@'%';
      GRANT UPDATE (review_count) ON yelp_db.business TO 'user2'@'%';
      GRANT UPDATE ON yelp_db.user TO 'user2'@'%';
    \end{verbatim}

\subsection{Group 3}

    Business analysts are special casual users. Here we assume they are not logged in so they are not expected to change any contents in the database. Therefore, we only add some view-related privileges to this groups of user besides those granted to group 1:

    \begin{verbatim}
      DROP USER IF EXISTS 'user3'@'%';
      CREATE USER user3;
      GRANT SELECT, CREATE VIEW, SHOW VIEW ON yelp_db.* TO 'user3'@'%';
    \end{verbatim}

\subsection{Group 4}

    Group 4 corresponds to normal developers. These people are in charge of the visiting, development and maintenance of database. Therefore we grant full IUD privileges on the whole database to them. Also, in case they need to perform automated operations, query optimization or concurrency control, we also grant them with view, routine(function, procedure), index and lock permissions. The SQL queries are as following:

    \begin{verbatim}
      DROP USER IF EXISTS 'user4'@'%';
      CREATE USER user4;
      GRANT ALTER ROUTINE, CREATE ROUTINE, EXECUTE, # routine related
      CREATE VIEW, SHOW VIEW, # view related
      CREATE, ALTER, INDEX, REFERENCES, # tables, indexes and keys
      DELETE, DROP, INSERT, SELECT, UPDATE # basic operations including IUD
      ON yelp_db.* TO 'user4'@'%';
    \end{verbatim}

\subsection{Group 5}
    
    Group 5 is the database administrator, so its privilege is all but \texttt{GRANT} and \texttt{PROXY} options, which should only be done using the root user. In practical use only these two operations should be done using root user in order to prevent abuse or unexpected threats to the database:

    \begin{verbatim}
      DROP USER IF EXISTS 'user5'@'%';
      CREATE USER user5;
      GRANT ALL ON yelp_db.* TO 'user5'@'%';
    \end{verbatim}

% COPY END
\end{document}
