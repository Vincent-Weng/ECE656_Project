\documentclass[12pt]{scrbook}

\setkomafont{author}{\scshape}
\linespread{1}
\usepackage[margin=1in]{geometry} % 1 inch margins all around
\usepackage{graphicx} 

\title{ECE656 Project Report}
\subtitle{Finding spam comments within the Yelp dataset}
\date{March 2018}
\author{Josh Reid\thanks{University of Waterloo}
\and Vincent Weng\footnotemark[1]}

\begin{document}
\maketitle
\section{Introduction}
Every year Yelp releases their extensive dataset to the public to analyse and try to find interesting
trends or draw new conclusions about their users or businesses listed on there.
This provides us an ideal dataset to study and analyse for ECE656 since it is such a large and diverse
dataset it means that there are several different conclusions that can be found and some potential
irregularities in their database structure that we can improve upon.
This report is broken up into two different sections, in the first section we analyse the structure
of the dataset to create an E-R diagram outlining the relation between all of the tables, clean the data 
and finally index the data to allow for efficient querying of the data.


\end{document}
